
\documentclass[letterpaper,11pt]{article}
%%%%%%%%%%%%%%%%%%%%%%%%%%%%%%%%%%%%%%%%%%%%%%%%%%%%%%%%%%%%%%%%%%%%%%%%%%%%%%%%%%%%%%%%%%%%%%%%%%%%%%%%%%%%%%%%%%%%%%%%%%%%%%%%%%%%%%%%%%%%%%%%%%%%%%%%%%%%%%%%%%%%%%%%%%%%%%%%%%%%%%%%%%%%%%%%%%%%%%%%%%%%%%%%%%%%%%%%%%%%%%%%%%%%%%%%%%%%%%%%%%%%%%%%%%%%
\usepackage{graphicx}
\usepackage{amsmath,amsfonts,amssymb,amsthm,float}
\usepackage{hyperref}
\usepackage[utf8]{inputenc}
\usepackage[left=2cm, right=2cm, top=2cm, bottom=2cm]{geometry}

\setcounter{MaxMatrixCols}{10}
%TCIDATA{OutputFilter=LATEX.DLL}
%TCIDATA{Version=5.50.0.2953}
%TCIDATA{<META NAME="SaveForMode" CONTENT="1">}
%TCIDATA{BibliographyScheme=BibTeX}
%TCIDATA{LastRevised=Tuesday, December 10, 2024 23:02:01}
%TCIDATA{<META NAME="GraphicsSave" CONTENT="32">}
%TCIDATA{ComputeDefs=
%$Z=0.033$
%$C=1.5$
%$R=0.95$
%$L=0.01$
%}


\input{tcilatex}
\renewcommand{\baselinestretch}{1.15}
\setlength{\parindent}{0pt}
\setlength{\parskip}{0.5\baselineskip}
\pretolerance=2000 \tolerance=3000
\renewcommand{\abstractname}{Resumen}

\begin{document}

\title{Proyecto Final: Sistema de transporte de ox\'{\i}geno en la sangre \
"SIRA"}
\author{Bejarano Lozada Elias \\
%EndAName
Departamento de Ingenier\'{\i}a El\'{e}ctrica y Electr\'{o}nica\\
Tecnol\'{o}gico Nacional de M\'{e}xico / Instituto Tecnol\'{o}gico de Tijuana%
}
\maketitle

\noindent \noindent \textbf{Palabras clave: }Controlador PID, Tratamiento, 

Correo:

\noindent \noindent Carrera: \textbf{Ingenier\'{\i}a Biom\'{e}dica}

\noindent Asignatura: \textbf{Modelado de Sistemas Fisiol\'{o}gicos}

\noindent Profesor: \href{https://biomath.xyz/}{\textbf{Dr. Paul Antonio
Valle Trujillo}} (paul.valle@tectijuana.edu.mx)

\section{Funci\'{o}n de transferencia}

Para nuestra funci\'{o}n de transferencia, asignamos distintas variables
para simplificar nuestra ecuaci\'{o}n:

$\ \ \ \ \ a=LC_{1}C_{2}R_{2}$

$\ \ \ \ \ b=C_{1}C_{2}R_{1}R_{2}+LC_{1}+LC_{2}$

$\ \ \ \ \ c=C_{1}R_{1}+C_{2}R_{1}+C_{2}R_{2}$

$\ \ \ \ \ d=1$

\begin{eqnarray*}
\frac{P_{p}\left( s\right) }{P_{ao}\left( s\right) } &=&\frac{\left( \frac{1%
}{C_{2}s}\right) I_{2}\left( s\right) }{\left( \frac{as^{3}+bs^{2}+cs+d}{%
sC_{2}}\right) I_{2}\left( s\right) } \\
\frac{P_{p}\left( s\right) }{P_{ao}\left( s\right) } &=&\frac{\frac{1}{C_{2}s%
}}{\frac{as^{3}+bs^{2}+cs+d}{sC_{2}}} \\
\frac{P_{p}\left( s\right) }{P_{ao}\left( s\right) } &=&\frac{\left(
1\right) \left( sC_{2}\right) }{\left( as^{3}+bs^{2}+cs+d\right) \left(
C_{2}s\right) } \\
\frac{P_{p}\left( s\right) }{P_{ao}\left( s\right) } &=&\frac{1}{%
as^{3}+bs^{2}+cs+d}
\end{eqnarray*}

\subsection{Ecuaciones principales}

\bigskip\ Ecuaci\'{o}n de voltaje de entrada, igualdad de voltaje y voltaje
de salida.

\begin{eqnarray*}
P_{ao}\left( t\right) &=&L\frac{di_{1}\left( t\right) }{dt}+R_{1}i_{1}\left(
t\right) +\frac{1}{C_{1}}\int \left[ i_{1}\left( t\right) -i_{2}\left(
t\right) \right] dt \\
\frac{1}{C_{1}}\int \left[ i_{1}\left( t\right) -i_{2}\left( t\right) \right]
dt &=&R_{2}i_{2}\left( t\right) +\frac{1}{C_{2}}\int i_{2}\left( t\right) dt
\\
P_{p}\left( t\right) &=&\frac{1}{C_{2}}\int i_{2}\left( t\right) dt
\end{eqnarray*}

\subsection{Ecuaciones integrodiferenciales}

\bigskip Despejamos $i_{1}$ 
\begin{eqnarray*}
P_{ao}\left( t\right) &=&L\frac{di_{1}\left( t\right) }{dt}+R_{1}i_{1}\left(
t\right) +\frac{1}{C_{1}}\int \left[ i_{1}\left( t\right) -i_{2}\left(
t\right) \right] dt \\
R_{1}i_{1}\left( t\right) &=&V_{e}\left( t\right) -L\frac{di_{1}\left(
t\right) }{dt}-\frac{1}{C_{1}}\int \left[ i_{1}\left( t\right) -i_{2}\left(
t\right) \right] dt \\
i_{1}\left( t\right) &=&\left[ P_{ao}\left( t\right) -L\frac{di_{1}\left(
t\right) }{dt}-\frac{1}{C_{1}}\int \left[ i_{1}\left( t\right) -i_{2}\left(
t\right) \right] dt\right] \frac{1}{R_{1}}
\end{eqnarray*}

\bigskip \bigskip Despejamos $i_{2}$

\begin{eqnarray*}
\frac{1}{C_{1}}\int \left[ i_{1}\left( t\right) -i_{2}\left( t\right) \right]
dt &=&R_{2}i_{2}\left( t\right) +\frac{1}{C_{2}}\int i_{2}\left( t\right) dt
\\
R_{2}i_{2}\left( t\right) &=&\frac{1}{C_{1}}\int \left[ i_{1}\left( t\right)
-i_{2}\left( t\right) \right] dt-\frac{1}{C_{2}}\int i_{2}\left( t\right) dt
\\
i_{2}\left( t\right) &=&\left[ \frac{1}{C_{1}}\int \left[ i_{1}\left(
t\right) -i_{2}\left( t\right) \right] dt-\frac{1}{C_{2}}\int i_{2}\left(
t\right) dt\right] \frac{1}{R_{2}}
\end{eqnarray*}

\bigskip

\begin{equation*}
P_{p}\left( t\right) =\frac{1}{C_{2}}\int i_{2}\left( t\right) dt
\end{equation*}

\subsection{Transformada de Laplace}

\bigskip 
\begin{eqnarray*}
P_{ao}\left( s\right) &=&LsI_{1}\left( s\right) +R_{1}I_{1}\left( s\right) +%
\frac{I_{1}\left( s\right) -I_{2}\left( s\right) }{C_{1}s} \\
\frac{1}{C_{1}s}\left[ I_{1}\left( s\right) -I_{2}\left( s\right) \right]
&=&\left( R_{2}+\frac{1}{C_{2}s}\right) I_{2}\left( s\right) \\
P_{p}\left( s\right) &=&\frac{I_{2}\left( s\right) }{C_{2}s}
\end{eqnarray*}

\subsection{Procedimiento algebraico}

\begin{eqnarray*}
\frac{1}{C_{1}s}\left[ I_{1}\left( s\right) -I_{2}\left( s\right) \right]
&=&\left( R_{2}+\frac{1}{C_{2}s}\right) I_{2}\left( s\right) \\
\frac{I_{1}\left( s\right) }{C_{1}s}-\frac{I_{2}\left( s\right) }{C_{1}s}
&=&\left( R_{2}+\frac{1}{C_{2}s}\right) I_{2}\left( s\right) \\
\frac{I_{1}\left( s\right) }{C_{1}s} &=&\left( R_{2}+\frac{1}{C_{2}s}\right)
I_{2}\left( s\right) +\frac{I_{2}\left( s\right) }{C_{1}s} \\
\frac{1}{C_{1}s}I_{1}\left( s\right) &=&\left( R_{2}+\frac{1}{C_{2}s}+\frac{1%
}{C_{1}s}\right) I_{2}\left( s\right) \\
I_{1}\left( s\right) &=&\frac{\left( R_{2}+\frac{1}{C_{2}s}+\frac{1}{C_{1}s}%
\right) I_{2}\left( s\right) }{\frac{1}{C_{1}s}} \\
I_{1}\left( s\right) &=&\left( \frac{R_{2}+\frac{1}{C_{2}s}+\frac{1}{C_{1}s}%
}{\frac{1}{C_{1}s}}\right) I_{2}\left( s\right) \\
I_{1}\left( s\right) &=&\left[ \frac{\frac{1}{sC_{1}C_{2}}\left(
C_{1}+C_{2}+R_{2}sC_{1}C_{2}\right) }{\frac{1}{C_{1}s}}\right] I_{2}\left(
s\right) \\
I_{1}\left( s\right) &=&\frac{C_{1}+C_{2}+R_{2}sC_{1}C_{2}}{C_{2}}%
I_{2}\left( s\right)
\end{eqnarray*}

\begin{equation*}
P_{p}\left( s\right) =\frac{I_{2}\left( s\right) }{C_{2}s}
\end{equation*}

\textbf{Sustituir }$I_{1}\left( s\right) $\ \textbf{en} $P_{ao}\left(
s\right) $:

\begin{eqnarray*}
P_{ao}\left( s\right) &=&LsI_{1}\left( s\right) +R_{1}I_{1}\left( s\right) +%
\frac{I_{1}\left( s\right) -I_{2}\left( s\right) }{C_{1}s} \\
P_{ao}\left( s\right) &=&LsI_{1}\left( s\right) +R_{1}I_{1}\left( s\right) +%
\frac{I_{1}\left( s\right) }{C_{1}s}-\frac{I_{2}\left( s\right) }{C_{1}s} \\
P_{ao}\left( s\right) &=&\left( Ls+R_{1}+\frac{1}{C_{1}s}\right) I_{1}\left(
s\right) -\frac{I_{2}\left( s\right) }{C_{1}s} \\
P_{ao}\left( s\right) &=&\left( Ls+R_{1}+\frac{1}{C_{1}s}\right) \left( 
\frac{C_{1}+C_{2}+R_{2}sC_{1}C_{2}}{C_{2}}I_{2}\left( s\right) \right) -%
\frac{I_{2}\left( s\right) }{C_{1}s} \\
P_{ao}\left( s\right) &=&\left[ \left( Ls+R_{1}+\frac{1}{C_{1}s}\right)
\left( \frac{C_{1}+C_{2}+R_{2}sC_{1}C_{2}}{C_{2}}\right) -\frac{1}{C_{1}s}%
\right] I_{2}\left( s\right) \\
P_{ao}\left( s\right) &=&\left[ \frac{s^{3}LC_{1}C_{2}R_{2}+\left(
C_{1}C_{2}R_{1}R_{2}+LC_{1}+LC_{2}\right) s^{2}+\left(
C_{1}R_{1}+C_{2}R_{1}+C_{2}R_{2}\right) s+1}{sC_{2}}\right] I_{2}\left(
s\right)
\end{eqnarray*}

$\ \ \ \ \ a=LC_{1}C_{2}R_{2}$

$\ \ \ \ \ b=C_{1}C_{2}R_{1}R_{2}+LC_{1}+LC_{2}$

$\ \ \ \ \ c=C_{1}R_{1}+C_{2}R_{1}+C_{2}R_{2}$

$\ \ \ \ \ d=1$

\begin{equation*}
P_{ao}\left( s\right) =\left( \frac{as^{3}+bs^{2}+cs+d}{sC_{2}}\right)
I_{2}\left( s\right)
\end{equation*}

\bigskip

\section{Error en estado estacionario}

\begin{eqnarray*}
e\left( t\right) &=&\lim_{s\rightarrow 0}\left( \frac{1}{s}\right) \left( 1-%
\frac{P_{p}\left( s\right) }{P_{ao}\left( s\right) }\right) \\
e\left( t\right) &=&\lim_{s\rightarrow 0}\left( \frac{1}{s}\right) \left( 1-%
\frac{1}{as^{3}+bs^{2}+cs+d}\right) =1-1=0
\end{eqnarray*}

\section{C\'{a}lculo de componentes para el controlador PID}

Con base en las ganancias sincronizadas con Simulink, las cuales estan dadas
por lo siguiente:%
\begin{eqnarray*}
k_{P} &=&7504.31283990786 \\
k_{I} &=&40028.7836532803 \\
k_{D} &=&312.586659285931
\end{eqnarray*}

$\,$

Entonces, para realizar el c\'{a}lculo de los valores de los componentes, se
propone un valor de capacitancia para $C_{r}$, con el cual se c\'{a}lcula un
valor para la resistencia $R_{e}$, ahora, siguiendo el mismo procedimiento
se calcula el valor de la resistencia $R_{r}$.

:

\begin{eqnarray*}
C_{r} &=&1\times 10^{-6} \\
R_{e} &=&\frac{1}{k_{I}C_{r}}=\frac{1}{\left( 40028.7836532803\right) \left(
1\times 10^{-6}\right) }=24.\,\allowbreak 982\Omega \\
R_{r} &=&k_{P}R_{e}=\left( 7504.31283990786\right) \left( 24.\,\allowbreak
982\right) =1.\,\allowbreak 874\,7\times 10^{5}\Omega \\
C_{e} &=&\frac{k_{D}}{R_{r}}=\frac{312.586659285931}{1.\,\allowbreak
874\,7\times 10^{5}}=1.\,\allowbreak 667\,4\times 10^{-3}F
\end{eqnarray*}

\end{document}
